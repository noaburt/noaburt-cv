\documentclass[a4paper]{moderncv}

\moderncvtheme[blue]{classic}
\usepackage{geometry}
\nopagenumbers{}

\name{Noa}{Burt}
\title{Curriculum Vitae}
\email{noaburt051221@gmail.com}
\phone[mobile]{(+44)~7422~523075}
\social[linkedin]{noaburt}
\social[github]{noaburt}

\begin{document}
\makecvtitle
\section{Education}
\cventry{2022--present}{BEng Computer \& Electronic Systems}{University of Strathclyde}{Glasgow, Scotland}{}{}
\cventry{2016--2022}{High School}{Johnstone High School}{Johnstone, Scotland}{}{\textbf{Results (5\textsuperscript{th} and 6\textsuperscript{th} Year):}%
\begin{itemize}%
    \item AH Computing Science (B);
    \item AH Mathematics (D);
    \item H Physics (A);
    \item H English (A);
    \item SCQF Level 5 Cyber Security (P);
\end{itemize}}

\section{Experience}
\subsection{Curricular}
\cventry{2023--2024}{FPGA Design Project}{University of Strathclyde}{}{}{2\textsuperscript{nd} Year VHDL Class: Group Assignment\newline{}\newline{}
One of the 2\textsuperscript{nd} year modules as part of the CES course at Strathclyde is centered around VHDL programming and FPGA board implementation. For the final assignment for this class, we were assigned a group design project; to design and implement a solution for a set task, collect our design process and implementation in a professional report, and demonstrate our solution with a reasonable understanding of the implementation.\newline{}\newline{}
This class was my first introduction to such FPGA programming, while also allowing me to develop further my team working and report writing skills. The limitations present due to working with the nature of describing hardware in VHDL provided very challenging at first, but these challenges sparked my intrest in FPGA development, and led to me seeking out such a summer internship for 2025.\newline{}
}
\subsection{Extra-Curricular}
\cventry{2023--present}{Formula Student FS-AI}{University of Strathclyde Motorsport}{}{}{Active member of the University of Strathclyde's Formula Student team \textbf{USM}, specifically the Formula Student AI (FS-AI) team.\newline{}\newline{}
In 2023/24, my first year in the team, I played a roll in leading the team to compete the the Formula Student UK Driverless competition for the first time in the team's history. This allowed me to improve many skills valuable to working in industry that my course otherwise would not.\newline{}\newline{}
Development of our software involved many tools and languages I had previously been unexperienced with, such as:
\begin{itemize}
    \item ROS2 (Robot Operating System 2) Framework
    \item CAN Interfacing in C++
    \item Docker Containers
    \item Interconnected Software Systems
    \item Foxglove \& Dynamic Bicycle Physics Simulation\newline{}
\end{itemize}
Then, moving into 2024/25, I led the development of a new automated testing system for our software, which utilised many new tools avaliable to the team that year, such as GitLab's CI and Cloud Hosting systems.
}
\cventry{2025}{FPGA Development Summer Internship}{Thales}{Govan, Scotland}{}{Still doing}
\cventry{2022--2025}{Kitchen Porter}{The Boarding House}{Howwood, Scotland}{}{Member of kitchen staff for a popular Gastropub in Howwood.\newline{}Employment here has developed my time management skills, as well as often dealing with responsibility for customer food presentation in a fast-paced environment.}
\cventry{2016--2024}{Bagpiper}{Johnstone Pipe Band}{Johnstone, Scotland}{}{Playing member of Johnstone Pipe Band (JPB).\newline{}Playing for the Renfrewshire Schools Pipe Band, the novice branch of JPB at the time, before being promoted straight to the top level band (Grade 1) has taught me crucial skills for being a good team member, responsibility for putting in work in my own time, and general dicipline with the instrument as well as personal appearance}

\end{document}